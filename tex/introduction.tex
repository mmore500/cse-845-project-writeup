\section{Introduction}

Stigmergy is an indirect mechanism of coordination, where an action is left on the environment to stimulate the performance of a future action \cite{susi_social_2001, ahadeli_multi-agent_2004}.
Pierre-Paul Grassét created this idea in 1959 when studying termite social behavior \cite{heylighen_stigmergy_2016a}.
Most often stigmergy is studied in a group setting, but can also be important in studying self- organizing behavior \cite{heylighen_stigmergy_2016b}.
Stigmergy is most attributed to insect colony migration and food savaging.
This type of stigmergy evolved the use of marker-based stigmergy \cite{ahadeli_multi-agent_2004, heylighen_stigmergy_2016b}, where a pheromone is left on the environment to stimulate an action from another member of the colony or self.
Marker based stigmergy is the most common stigmergy used in social insect groups because of the evolved common knowledge of released pheromones.
But in some environments, such as a desert, there is a high pheromone volatility, which results in the evolution of the Sematectonic stigmergy where the environment or a change of environment stimulates a response \cite{ahadeli_multi-agent_2004, heylighen_stigmergy_2016a}.
Due to the high volatility, the organisms choose use their environment to stimulate a reaction rather then placing down a pheromone.
A main factor that stimulates the production of Sematectonic stigmergy agents is nest construction. This allows the workers to systematically move building pillars during the later stage of nest construction \cite{dorigo_ant_2000,khuong_stigmergic_2016}.

Additionally, another factor the organism has control over and evolved the use of is the amount of stigmergy pheromones released into the environment and the pheromone’s relative strength.
Stigmergic organisms have a choice in the strength and the kind of pheromone that is released \cite{theraulaz1999brief}.
When the change in strength relates to a different signal, such as location to the food source this relates to quantitative stigmergy.
If the difference of the response is directly related to the change in type of pheromones, then this results in Qualitative stigmergy \cite{heylighen_stigmergy_2016b}.
This is seen in the ant species \textit{Monomorium pharaonis} where there is the production of both an attractive and repellent pheromone based on the desired outcome of the group \cite{jeanson_pheromone_2003}.
Stigmergy is not only used in ant and insect models, it is also used in many other organisms.
Bacteria are also an organism that has shown to have a wide use of stigmergy.
Biofilm formation is a key role in bacterial survival and due to this the bacteria has evolved the use of stigmergy \cite{gloag_bacterial_2015}.
A biofilm is a attachment of organisms together on a surface.
Often biofilms are associated with higher antibiotic resistance and survival \cite{donlan2002biofilms}.
\textit{Pseudomonas aeruginosa} are able to move in the environment by twitching motility and the type IV pili \cite{persat2015type}.
This motility uses the type IV pili that creates channels as they move during the production of the biofilm.
This allows other bacteria to follow and aide in the production of the biofilm \cite{gloag_stigmergy:_2013}.

A major reason of the widespread use of stigmergy in ant populations is the volatility of the released pheromone.
Since the environment and the food sources are not always a constant factor, the ants have evolved the use of different volatile pheromones in relationship to the stability of the target \cite{jeanson_pheromone_2003}.
In the ant colonies of \textit{Atta columbica} their released pheromones can persist for many years \cite{howard_costs_2001}.
This long half-life is contributed to the constant availability of a high quality food source.
Alternatively, the \textit{Monomorium pharaonis} is able to produce a short half-life pheromone that decays within minutes of production, which guides worker ants to the food source currently available \cite{robinson_decay_2008}.
Since the stability of the food source is relatively low in these ant environments, a constant replenishing of the pheromone is required.
It could be concluded that the volatility of the pheromones is in direct relationship to the stability of the food sources.
The environment also plays a role in the amount of stigmergic pheromones are placed in the environment.
It has been shown that the unstable and changing the environment is the more pheromones the ants will place down \cite{czaczkes2015trail}.
Another factor that the environment plays in the use of stigmergy alone or in a group.
It has been shown that desert ant do not use pheromones to recruit workers as group rather because of the high temperatures evolved the use of pheromones for foraging on their own \cite{ruano_high_2000}.

There are many factors that contribute to the evolution of stigmergic strategies.
It can be directly related to the difficulty of an agent to navigate their specific environment.
Furthermore, the inability for the agent to form internal representation of their environment has a direct influence on the evolution of stigmergic strategies.

An abundance of mathematical and computational models of stigmergy-based navigation have been developed and studied extensively \cite{perna2012individual, ryan2016model}.
Numerical evaluation of differential equation models have yielded emergent patterns closely resembling the colony-level foraging behaviors of ants.
Such models are generally parameterized using values drawn directly from biological observations.
Computational agent-based models with manually-designed agent algorithms have also yielded emergent patterns closely resembling the colony-level foraging behaviors of ants \cite{robinson2008agent, pratt2005agent}.
Many agent-based models fall under the umbrella of Ant Colony Optimization (ACO) \cite{dorigo1996ant}, an applied problem-solving technique used to tackle challenges like the traveling salesman problem \cite{dorigo1997ant, bianchi2002ant}.
Some work has been done using evolution-inspired techniques like evolutionary strategies and the genetic algorithm to optimize parameters, including pheromone evaporation rate, of these models \cite{kuyucu2012evolutionary, sauter2002evolving}.
Self-adaptive methods of pheromone evaporation rate optimization have also been developed \cite{mavrovouniotis2014ant, mavrovouniotis2013adapting}.
Panait et al. explored how pheromone diffusion and evaporation rate affected amount of food items foraged in their model, finding that intermediate pheromone diffusion rates maximized foraging performance and that foraging performance decreased at high pheromone evaporation rates \cite{panait2004ant}.

Still other computational agent-based models evolve agent behavior in a more open-ended fashion using evolving artificial neural networks (EANN) \cite{collins1991antfarm} or genetic programming \cite{connelly2009evolving}, providing a unique opportunity to investigate questions about the evolution of stigmergy-based navigation behavior.
However, to our knowledge work has not been done to systematically explore pheromone evaporation rate on behavior in such open-ended evolving systems.

The pheromone evaporation rate is of great interest in the context of comparisons between collective behavior mediated by stigmergy and cognition, providing an analogy to short-term versus long-term memory and harboring implications with respect to to the speed-accuracy trade-off \cite{couzin2009collective, correia2017role}.
As such, understanding the evolutionary implications of stigmergic volatility are pertinent to questions of the evolution of intelligence.
Less abstractly, it is also of interest to understanding the relationship between pheromone volatility and food source permanence \cite{howard_costs_2001,robinson_decay_2008}.

We set out to investigate how pheromone evaporation rate affects the effectiveness of evolved navigation strategies and the extent to which evolved navigation strategies depend on stigmergy.
% We are interested in this question in relation to the permanence of the foraging food source.
