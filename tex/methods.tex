\section{Methods}

\subsection{MABE}

In this work we use the Modular Agent-Based Evolver\cite{bohm_mabe_2017}  (MABE) framework to build and run our experiments. MABE allows for the quick construction of agent-based evolution simulations by allowing the scientist to reuse aspects of the system that have already been built by other members of the MABE community. In our work, we use the modules for Markov Brains, Circular Haploid Genomes, simple mutation-only reproduction via roulette selection, and population data archiving. Our contribution is therefore minimized to simply creating the world module, the module that defines the environment the agent will be evaluated in.

\subsection{Stigmergy World}

The world module is responsible for defining three key aspects of the simulation: The environment, how the agent senses the environment, and how the agent can change the environment.

In this work, the environment is an $n\times m$ grid surrounded by a wall, and filled with smaller wall segments that create obstructions for the agent to navigate around. In addition, the environment contains one home location, where the agent will begin each simulation, and a food location. The environment is constructed by first generating a maze and then removing walls until the desired density of obstacles is reached(Figure \ref{fig:world_explanatory}). This method was chosen to ensure that home and food locations are always reachable by some path through the obstacles.

\begin{figure}
\begin{center}
\includegraphics[width=\textwidth]{img/world_explanatory}
\caption{
Caption TODO
}
\label{fig:world_explanatory}
\end{center}
\end{figure}


The maze layout, wall removal, home location, and initial food location are all generated randomly during environment construction. A new environment is created before every agent evaluation. Furthermore, the location of food may change during an agent's lifetime according to the rules for food use. When the food must move, it is randomly relocated.
The agent is equiped with a total of $43+k$ bits of sensor data (Figure \ref{fig:sense_explanatory}). The agent has a vision cone that covers 8 cells with the capability of distinguishing 4 environmental states at each location (32 bits), a $3\times 3$ stigmergy proximity sensor centered on the agent's location (9 bits), an internal compass that provides a two bit representation of the four cardinal directions, and a stigmergy read sensor that senses  the $k$ bit stigmergy signal. In this work we have set $k$ to be $1$, disallowing the agent access to multiple, distinguisable, stigmergy signals. The agent recieves input from each of its sensors at every world update.

\begin{figure}
\begin{center}
\includegraphics[width=\textwidth]{img/sense_explanatory}
\caption{
The agent's sensors' ranges of influence are indicated by the shaded locations. Left to Right: Agent vision cone wall, food, and home components; Agent pheromone proximity sensor; Agent pheromone secretion and discrimination location; and agent compass.
}
\label{fig:sense_explanatory}
\end{center}
\end{figure}


The agent has a total of $2+k$ output bits. The agent moves throughout the environment via tank controlls (Figure \ref{fig:movement_explanatory}) (2 bits). The agent can make no movment by outputing $00$ to the movment bits. The agent also writes $k$ bits to the current location in the form of a stigmergy signal. Writing all zeros is the same as writing no signal. The agnet will always output signal on every world update, however it can choose to do nothing by writing zeros to the move bits, stigmergy bits, or all output bits. Recall that we have set $k$ to be $1$ in this work, so the agent can only write a single kind of signal or write no signal.

\begin{figure}[!htbp]
\begin{center}
\includegraphics[width=\textwidth]{img/movement_explanatory}
\caption{
Agent locomotion: Image is to be read from bottom to top, with time t at the bottom and time t+1 at the top. \textbf{Outer Left:} Agent moves forward one cell via setting its locomotion bits to '11'. \textbf{Inner Left:} Agent Turns left via '10'. \textbf{Inner Right:} Agent turns right via '01'. \textbf{Outer Right:} Agent may remain stationary by outputing '00'.
}
\label{fig:movement_explanatory}
\end{center}
\end{figure}


\subsection{Experimental Design}

ba

