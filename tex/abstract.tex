\begin{abstract}
% Abstract length should not exceed 250 words
%One or two sentences providing a basic introduction to the field, comprehensible to a scientist in any discipline
Stigmergy is an indirect mechanism of coordination where an action alters the environment and the changed environment, in turn, alters the likelihood of future actions.
Pheromone-marker-based stigmergy is the most common form of stigmergy used in social insect groups.
%Two to three sentences of more detailed background, comprehensible to scientists in related disciplines
The ant species \textit{Monomorium pharaonis} produces pheromones that decay after several minutes, whereas \textit{Atta columbica} ants produce pheromones that persist for several years.
The respective transience and permanence of these ants' foraging paths is thought to be related to their pheromones' volatility.
Several mathematical and computational models have demonstrated that relatively moderate pheromone decay rates are correlated with increased foraging success in stigmergic agents.
%One sentence clearly stating the general problem being addressed by this particular study
However, the implications of pheromone volatiltiy on the evolution of navigation behavior are not well-studied.
Here we use digital agent-based evolution to show that moderate pheromone decay rates correspond to an increase in the extent to which agent foraging behavior relies on stigmergic cues.
This result suggests that the properties of an organism's pheromones play a vital role in the organism's evolutionary trajectory, driving some towards, and others away from, stigmergic foraging strategies.
These results help shed light on how elements of an organism's environment, such as the stability of its food source, relate to the properties of that organism's pheromones.
Understanding the evolution of stigmergic group coordination strategies, an instance of collective cognition, provides a foundation for addressing the larger questions surrounding the evolution of intelligence.

\end{abstract}
