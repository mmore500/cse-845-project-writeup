\begin{abstract}
% Abstract length should not exceed 250 words
%One or two sentences providing a basic introduction to the field, comprehensible to a scientist in any discipline
Stigmergy is an indirect mechanism of coordination where an action alters the environment and the changed environment, in turn, alters the likelihood of future actions.
Pheromone, marker-based, stigmergy is the most common form of stigmergy used in social insect groups.
%Two to three sentences of more detailed background, comprehensible to scientists in related disciplines
The ant species \textit{Monomorium pharaonis} is known to produce both an attractive and repellent pheromone marker allowing for both positive and negative feedback.
Additionally,\textit{Monomorium pharaonis} produces pheromones that decay after several minutes, whereas \textit{Atta columbica} produce pheromones that persist for several years.
It has been hypothesized that the vast range of decay rates may be linked to the stability of the species’ food source.
%One sentence clearly stating the general problem being addressed by this particular study
However, the precise environmental conditions necessary to drive evolution toward the use of stigmergy as a form of navigational coordination is unknown.
%One sentence summarizing the main result (with the words “here we show” or their equivalent).
Here we use digital agent-based evolution to show that a moderate pheromone decay rate corresponds with an increase in likelihood that an agent evolves a stigmergic foraging strategy.
%Two or three sentences explaining what the main result reveals in direct comparison to what was thought to be the case previously, or how the main result adds to previous knowledge.
X
X
X
%One or two sentences to put the results into a more general context
Y
Y
%Two or three sentences to provide a broader perspective, readily comprehensible to a scientist in any discipline
Z
Z
Z
\end{abstract}
