\section{Conclusion}

Our observation of maximal foraging performance at intermediate stigmergy evaporation rates qualitatively concurs with results from existing models with manually-designed navigation strategies \cite{panait2004ant}.
Interestingly, the evolutionary component of our model suggests that although evolved behavior cannot fully overcome the limitations of too-slow or too-fast stigmergy evaporation, evolved navigation strategies appear to compensate by relying less heavily on stigmergic cues.
This result shows that the extent to which stigmergic cues govern evolved foraging behavior can vary progressively.
This suggests, perhaps, that transitions to or from behavioral reliance on stigmergic cues need not be viewed as a on-or-off leap.

In future work, we will investigate the relationship between stigmergic volatility and food source permanence noted in \textit{Monomorium pharaonis} and \textit{Atta columbica} \cite{jeanson_pheromone_2003, howard_costs_2001, robinson_decay_2008}.
We expect to observe greater evolved foraging performance is achieved at low stigmergy volatility for permanent food placement and whether greater evolved foraging performance is observed at high stigmergy volatility for transient food placement.
These experiments will help confirm the functional implications of stigmergy volatility with respect to ask switching suggested by biological observation and shed light on the evolutionary roots of these differences in stigmergy volatility.
